\documentclass[14pt, b4paper]{article}
\usepackage{cmap}
\usepackage{fontspec}
\usepackage[T2A]{fontenc}
\usepackage[utf8x]{inputenc}
\usepackage[english,russian]{babel}
\usepackage{amsfonts}
\usepackage{amsmath}
\usepackage{amssymb}
\usepackage{amsthm}
\usepackage{graphicx}
\usepackage{listings}
\usepackage{color}

\graphicspath{./images/}

\defaultfontfeatures{Ligatures={TeX},Renderer=Basic} 
\setmainfont[Ligatures={TeX,Historic}]{Times New Roman}
\setmonofont[Ligatures={TeX,Historic}]{Times New Roman}


\usepackage{geometry} % Меняем поля страницы


\definecolor{dkgreen}{rgb}{0,0.6,0}
\definecolor{gray}{rgb}{0.5,0.5,0.5}
\definecolor{mauve}{rgb}{0.58,0,0.82}


\begin{document}
	\title{Листинг программы по теме "Классы"}
	\author{Владимир Татаринов}
	\date{\today}
	\maketitle
	\lstset{ %
		language=C++,                % Язык программирования 
		numbers=left,                   % С какой стороны нумеровать
		numberstyle=\color{gray},     % Стиль который будет использоваться для нумерации строк
		stepnumber=1,                   % Шаг между линиями. Если 1, то будет пронумерована каждая строка 
		numbersep=5pt,                  
		backgroundcolor=\color{white},      % Цвет подложки. Вы должны добавить пакет color - usepackage{color}
		showspaces=false,               
		showstringspaces=false,         
		showtabs=false,                
		frame=single,                    % Добавить рамку
		rulecolor=\color{black},        
		tabsize=2,                       % Tab - 2 пробела
		breaklines=true,                 % Автоматический перенос строк
		breakatwhitespace=true,          % Переносить строки по словам
		title=Программа на C++,                   % Показать название подгружаемого файла
		keywordstyle=\color{blue},          % Стиль ключевых слов
		commentstyle=\color{dkgreen},       % Стиль комментариев
		stringstyle=\color{mauve}          % Стиль литералов
	}
	
	\begin{lstlisting}
		#include <iostream>
		
		using namespace std;
		
		class ratio
		{
			private:
				int num, den;
				void reduce();
				int gcd(int n, int m);
				int lcm(int a, int b);
			public:
				ratio();
				ratio(int n, int d);
				void set(int, int);
				void show();
				ratio operator*(ratio);
				ratio operator+(ratio);
				ratio operator*(int);
				friend ratio operator*(int, ratio);
		};
		
		void ratio::set(int n, int d) {
			num = n;
			den = d;
			reduce();
		}
		
		ratio::ratio() {
			set(1,1);
		}
		
		ratio::ratio(int n, int d) {
			set(n, d);
		}
		
		void ratio::show() {
			if (num < 0)
				cout << endl << "  "<< -num << endl << "- ###" << endl << "  " << den << endl;
			else
				cout << endl << num << endl << "###" << endl << den << endl;
		}
		
		int ratio::gcd (int a, int b) {
			return b ? gcd (b, a % b) : a;
		}
		
		int ratio::lcm (int a, int b) {
			return a / gcd (a, b) * b;
		}
		
		void ratio::reduce() {
			int buf = abs(gcd(num,den));
			num /= buf;
			den /= buf;
			if (den < 0) {
				num *= (-1);
				den *= (-1);
			}
		}
		
		ratio ratio::operator*(ratio r) {
			return ratio(num * r.num, den * r.den);
		}
		
		ratio ratio::operator+(ratio r) {
			int b = lcm(r.den, den);
			int a = num * b / den + r.num * b / r.den;
			return ratio(a,b);
		}
		
		ratio ratio::operator*(int k) {
			return ratio(k * num, den);
		}
		
		ratio operator*(int k, ratio r) {
			return r*k;
		}
		
		int main() {
			ratio r1(2,-3), r2(3,4);
			r1 = r1 * r2;
			r1.show();
			r1.set(1,-6);
			r2.set(-3,14);
			r1 = r1 + r2;
			r1.show();
			r1 = 6 * r1;
			r1.show();
			r2 = r2 * 3;
			r2.show();
		}
	\end{lstlisting}
\end{document}